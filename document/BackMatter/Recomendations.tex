\begin{recomendations}
    El presente trabajo sienta las bases para continuar desarrollando un corpus m\'as extenso y 
    representativo de textos anotados con atributos protegidos. Para ello, se recomienda 
    continuar con el proceso de anotaci\'on manual siguiendo la metodolog\'ia propuesta,
    incorporando nuevos textos al corpus. Esto permitir\'a incrementar el tama\~no del 
    corpus, permitiendo el entrenamiento y evaluaci\'on de modelos m\'as robustos y efectivos.
    
    Por otro lado, se recomienda prestar atenci\'on a las clases con baja representaci\'on en el 
    corpus actual, como \emph{Indian}, \emph{Arab} y \emph{Native American} para el atributo raza.
    A medida que se ampl\'ie el corpus se espera solventar esta limitaci\'on al incluir m\'as 
    instancias de dichas clases. En este sentido, se puede analizar la posibilidad de crear una 
    clase \emph{Other} que agrupe razas con muy baja frecuencia en los textos. Otra alternativa es 
    siempre que sea posible, combinar clases con poca representaci\'on junto a otras m\'as 
    frecuentes.

    Se sugiere seguir desarrollando el sistema de extracci\'on autom\'atica, probando nuevos modelos
    y t\'ecnicas de procesamiento de lenguaje natural. El objeticvo es acercarse cada vez m\'as a 
    la habilidad humana en la tarea de anotaci\'on.

\end{recomendations}
