\chapter*{Introducción}\label{chapter:introduction}
\addcontentsline{toc}{chapter}{Introducción}
%ML everywhere
En la actualidad, los algoritmos de aprendizaje autom\'atico han adquirido significativa importancia, extendiendo su aplicaci\'on a 
diversas esferas de la vida. Estos algoritmos se han convertido en una herramienta fundamental para la toma de decisiones y la 
automatizaci\'on de tareas complejas. Entre las tareas m\'as destacadas se encuentran: sistemas de recomendaci\'on en 
plataformas~\parencite{esmaeilzadeh2022abuse, bhattacharya2022augmenting}, facilitar compras en l\'inea, mejoras en la eficiencia 
de los sistemas de transporte~\parencite{autonomous_driving} y predicciones en \'areas como la salud~\parencite{roy2023machine} y 
las finanzas~\parencite{sen2021machine}.

%ML model has been proved to contain bias
Las computadoras poseen la capacidad de procesar y analizar extensos vol\'umenes de informaci\'on. Adem\'as, ellas pueden considerar 
m\'ultiples variables simult\'aneamente en un tiempo considerablemente menor al que le tomar\'ia a un ser humano. Estas caracter\'isticas 
hacen muy atractivo el uso de dichos algoritmos en beneficio de la sociedad. Sin embargo, un problema emergente en este campo es la 
existencia de sesgos e injusticias en las decisiones tomadas por estos algoritmos. Se ha evidenciado en numerosas ocasiones que algunos 
modelos de aprendizaje autom\'atico no muestran imparcialidad en sus predicciones. En cambio, se observa que dichos modelos tienden a 
favorecer a ciertos segmentos o grupos de la poblaci\'on~\parencite{survey}.

%It is important to detect and mitigate bias
Es de vital importancia el trabajo en la detecci\'on y mitigaci\'on de sesgos e injusticias debido a la creciente dependencia y 
utilidad de estos algoritmos. Sin esfuerzos en este an\'alisis, los sesgos pueden perpetuarse y amplificarse a medida 
que los algoritmos se utilizan y actualizan con el tiempo, llevando a resultados cada vez m\'as perjudiciales.

%There are several techniques to detect and mitigate bias
Entre las t\'ecnicas para la detecci\'on de sesgos se destacan las basadas en la equidad de grupos~\parencite{fairmodels}. 
En estas t\'ecnicas se identifican grupos de individuos que comparten uno o m\'as atributos ``protegidos''. Existe tambi\'en una 
t\'ecnica muy interesante que se basa en la anotaci\'on autom\'atica de atributos protegidos en los 
datasets~\parencite{soumah2023radar,dinan2020multidimensional,10.1007/978-3-031-35320-8_39}. Entre los atributos que se asocian 
con mayor frecuencia a grupos vulnerables se encuentran el g\'enero, la raza, la orientaci\'on sexual y la nacionalidad. 
Siguiendo esta l\'inea se han logrado detectar injusticias en sistemas de contrataci\'on~\parencite{examples_dis}, sistemas de 
seguros de salud~\parencite{examples_dis}, e incluso en modelos de reconocimiento de voz~\parencite{voice_bias}.

% Ante la necesidad de garantizar la imparcialidad y equidad en modelos de aprendizaje autom\'atico, se han desarrollado diversas 
% t\'ecnicas para mitigar los sesgos existentes. Entre ellas se encuentran: \textit{disparate impact remover} \parencite{dis_impact_rem} 
% y \textit{fairness through awareness} \parencite{fair_awareness}, utilizadas antes y despu\'es del procesamiento del algoritmo 
% respectivamente. Existe tambi\'en un enfoque muy interesante que se basa en la anotaci\'on autom\'atica de atributos protegidos en 
% los datasets \parencite{soumah2023radar,dinan2020multidimensional,10.1007/978-3-031-35320-8_39}.

\subsection*{Problem\'atica}
%Need for a corpus to develop and evaluate these techniques - existence of complex cases where what is needed is non-tabular data
Una caracter\'istica inherente a todas las t\'ecnicas para el an\'alisis de sesgos es la necesidad de datasets con 
atributos protegidos correctamente anotados y balanceados. Por ende, es importante disponer de ellos con el fin de 
asistir al desarrollo y evaluaci\'on de estas t\'ecnicas. No obstante, existen casos m\'as complejos donde un dataset que incluya 
datos tabulares y atributos protegidos no se adapta al problema en cuesti\'on. Como ejemplo de estos se tiene:  
el an\'alisis de sentimientos en textos, modelos de traducci\'on autom\'atica para lenguajes con diversidad 
de g\'enero, o la detecci\'on de emociones en im\'agenes.

%Datasets with non-tabular data and annotated protected attributes are scarce - solving this problem would be great
La carencia y complejidad asociada a la obtenci\'on de datasets que incorporen datos no tabulares y atributos protegidos constituye un 
desaf\'io de gran relevancia y discusi\'on en la literatura. La insuficiencia de recursos de este tipo impone restricciones 
significativas al desarrollo de investigaciones en el \'area. Es por eso que resolver este problema permitir\'ia un avance sustancial 
en el an\'alisis de sesgos en algoritmos de aprendizaje autom\'atico.

\section*{Objetivos}
Este trabajo propone como objetivo fundamental el dise\~no y validaci\'on de un corpus de datos no tabulares para
asistir en el desarrollo de t\'ecnicas destinadas a la mitigaci\'on de sesgos y la anotaci\'on autom\'atica de atributos protegidos. El contenido del corpus
es de tipo textual y de dominio general. Adem\'as, se garantiza que contenga entidades nombradas, sustantivos y pronombres
que hagan referencia a atributos protegidos. Estos atributos son el g\'enero y la raza.

Se proponen los siguientes objetivos espec\'ificos:
\begin{itemize}
    \item Consultar la literatura especializada en el an\'alisis de sesgos y las caracter\'isticas predominantes de los corpus en el estado del arte.
    \item Analizar las posibles alternativas encontradas en la literatura para identificar la variante a desarrollar.
    \item Dise\~nar un modelo de anotaci\'on de atributos protegidos y desarrollar herramientas para asistirlo.
    \item Anotar un corpus a partir del modelo dise\~nado.
    \item Construir un prototipo computacional para comprobar la eficacia del modelo de anotaci\'on y del corpus anotado.
    \item Evaluar marco experimental y arribar a conclusiones.
\end{itemize}

\section*{Contribuciones}
La realizaci\'on de esta tesis aporta contribuciones sustanciales en el \'ambito del an\'alisis de sesgos en modelos de aprendizaje 
autom\'atico. Al abordar la carencia de \emph{datasets} con datos no tabulares y atributos protegidos anotados, el corpus desarrollado podr\'a ser utilizado para:
\begin{itemize}
    \item Entrenar y evaluar modelos de extracci\'on autom\'atica de atributos protegidos en texto.
    \item Evaluar t\'ecnicas de mitigaci\'on de sesgos identidicadas en la literatura, mediante su aplicaci\'on en modelos entrenados con el corpus.
    \item Organizar competencias y retos internacionales de equidad algor\'itmica que utilicen el corpus como conjunto de evaluaci\'on.
\end{itemize}

%se brindar\'a la oportunidad
%de realizar investigaciones m\'as exhaustivas y aplicaciones pr\'acticas que %impactar\'an positivamente en la equidad y justicia de 
%los algoritmos de aprendizaje autom\'atico.


\section*{Organizaci\'on de la Tesis}
El contenido de la tesis se organiza de la siguiente forma. El Cap\'itulo 1 realiza una revisi\'on de la literatura y el estado del arte en temas relacionados con el sesgo en modelos de aprendizaje autom\'atico. Adem\'as se analizan los principales corpus utilizados en el an\'alisis de sesgos. Luego, el Cap\'itulo 2 describe la metodolog\'ia propuesta para la contrucci\'on de un corpus con anotaciones de atributos protegidos. Se describen el modelo de anotaci\'on y el proceso de etiquetado manual realizado. El Cap\'itulo 3 explica el marco experimental propuesto para evaluar la efectividad, tanto del corpus generado, como del modelo de anotaci\'on propuesto, y se analizan los resultados obtenidos. Finalmente se arriban a conclusiones y se discuten la l\'ineas de investigaci\'on futuras.  