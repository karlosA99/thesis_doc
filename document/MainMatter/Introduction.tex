\chapter*{Introducción}\label{chapter:introduction}
\addcontentsline{toc}{chapter}{Introducción}

En la actualidad, los algoritmos de aprendizaje autom\'atico han adquirido significativa importancia, extendiendo su aplicaci\'on a diversas 
esferas de la vida. Desde sistemas de recomendaci\'on en plataformas de streaming \parencite{esmaeilzadeh2022abuse, bhattacharya2022augmenting}, hasta 
realizar predicciones en \'areas como la salud \parencite{roy2023machine} o las finanzas \parencite{sen2021machine}, los algoritmos de 
aprendizaje autom\'atico se han convertido en una herramienta fundamental para la toma de desiciones y la automatizaci\'on de tareas complejas.

Las m\'aquinas poseen la capacidad de procesar y analizar extensos vol\'umenes de informaci\'on. Adem\'as, pueden considerar m\'ultiples variables
simult\'aneamente en un tiempo considerablemente menor al que le tomar\'ia a un ser humano. Estas caracter\'isticas hacen muy atractivo el uso de
dichos algoritmos en beneficio de la sociedad. Sin embargo, un problema emergente y de creciente preocupaci\'on en este campo es la existencia de 
sesgos e injusticias en las desiciones tomadas por estos algoritmos. Se ha evidenciado en numerosas ocasiones que algunos modelos de aprendizaje 
autom\'atico no muestran imparcialidad en sus predicciones. En cambio, se observa que dichos modelos tienden a favorecer a ciertos segmentos 
o grupos de la poblaci\'on \parencite{survey}.
 
Es de vital importancia el trabajo en la detecci\'on y mitigaci\'on de sesgos e injusticias debido a la creciente dependencia y utilidad
de estos algoritmos. Sin esfuerzos en la detecci\'on y mitigaci\'on, los sesgos pueden perpetuarse y amplificarse a medida que los algoritmos se 
utilizan y actualizan con el tiempo, llevando a resultados cada vez m\'as perjudiciales. 



\section*{Objetivos}
Este trabajo propone como objetivo fundamental el dise\~no y validaci\'on de un corpus de datos no tabulares para
asistir en el desarrollo y evaluaci\'on de t\'ecnicas destinadas a la mitigaci\'on de sesgos.

Se proponen los siguientes objetivos espec\'ificos:
\begin{itemize}
    \item Consultar la literatura especializada para identificar las t\'ecnicas de mitigaci\'on de sesgos y creaci\'on
    de corpus predominantes en el estado del arte.
    \item Analizar las posibles alternativas encontradas en la literatura para identificar la variante a desarrollar.
    \item Dise\~nar una propuesta propia de un corpus para asistir t\'ecnicas de mitigaci\'on de sesgos. 
    \item Construir un prototipo computacional para comprobar la eficacia de la propuesta.
    \item Evaluar marco experimental y arribar a conclusiones.
\end{itemize}

\section*{Contribuciones}


\section*{Organizaci\'on de la Tesis}
