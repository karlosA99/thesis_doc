\chapter*{Introducción}\label{chapter:introduction}
\addcontentsline{toc}{chapter}{Introducción}

En la actualidad, los algoritmos de aprendizaje autom\'atico han adquirido significativa importancia, extendiendo su aplicaci\'on a diversas esferas.
Desde sistemas de recomendaci\'on en plataformas de streaming \parencite{esmaeilzadeh2022abuse, bhattacharya2022augmenting}, hasta realizar predicciones en \'areas como la salud o las finanzas, los algoritmos de aprendizaje
autom\'atico se han convertido en una herramienta fundamental para la toma de decisiones y la automatizaci\'on de tareas complejas.


 

\section*{Objetivos}
Este trabajo propone como objetivo fundamental el dise\~no y validaci\'on de un corpus de datos no tabulares para
asistir en el desarrollo y evaluaci\'on de t\'ecnicas destinadas a la mitigaci\'on de sesgos.

Se proponen los siguientes objetivos espec\'ificos:
\begin{itemize}
    \item Consultar la literatura especializada para identificar las t\'ecnicas de mitigaci\'on de sesgos y creaci\'on
    de corpus predominantes en el estado del arte.
    \item Analizar las posibles alternativas encontradas en la literatura para identificar la variante a desarrollar.
    \item Dise\~nar una propuesta propia de un corpus para asistir t\'ecnicas de mitigaci\'on de sesgos. 
    \item Construir un prototipo computacional para comprobar la eficacia de la propuesta.
    \item Evaluar marco experimental y arribar a conclusiones.
\end{itemize}

\section*{Contribuciones}


\section*{Organizaci\'on de la Tesis}
