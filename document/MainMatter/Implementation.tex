\chapter{An\'alisis Experimental}\label{chapter:implementation}
En este cap\'itulo, se presenta el marco experimental dise\~nado para comprobar la efectividad del modelo de anotaci\'on 
descrito en el Cap\'itulo~\ref{chapter:proposal}, y la calidad del corpus generado. Se describen las configuraciones y 
par\'ametros utilizadas en el proceso de validaci\'on y evaluaci\'on. Adem\'as, se muestran los resultados obtenidos y 
se realiza un an\'alisis de los mismos.

\section{Marco Experimental}\label{section:experimental_framework}
El marco experimental dise\~nado tiene como objetivo evaluar la validez de la propuesta desarrollada. Se plantea 
una experimentaci\'on que permite comprobar si la soluci\'on cumple con los requisitos y restricciones establecidos inicialmente.
Para ello, se proporcionan especificaciones detalladas acerca del corpus utilizado y los escenarios de evaluaci\'on desarrollados.
Adem\'as, se describen los hiperpar\'ametros utilizados y el hardware empleado para la experimentaci\'on.

\subsection{Escenarios de Evaluaci\'on}
La experimentaci\'on realizada se divide en dos escenarios. El primero de ellos se centra en analizar la concordancia alcanzada entre 
los anotadores humanos durante el proceso de construcci\'on del corpus. El segundo escenario, se orienta a utilizar el corpus 
desarrollado para entrenar y evaluar la eficacia de un modelo de clasificaci\'on de g\'enero y raza. Esto tiene como objetivo determinar
la capacidad de un anotador autom\'atico en replicar de manera confiable las anotaciones realizadas por los anotadores humanos. 

\subsubsection{Escenario I: Concordancia entre Anotadores}
Este escenario tiene como objetivo examinar la concordancia entre anotadores planteada en la 
secci\'on~\ref{subsection:annotation_evaluation}. El escenario busca obtener una estimaci\'on confiable de la dificultad de la 
tarea de anotaci\'on, as\'i como evaluar la calidad global de las anotaciones obtenidas. Esto se logra mediante la comparaci\'on
de las anotaciones realizadas de forma independiente por los dos anotadores no expertos, adem\'as de la comparaci\'on con la 
versi\'on final del corpus.

El an\'alisis de m\'etricas de concordancia inter-anotador permite determinar si las directrices propuestas en la 
secci\'on~\ref{section:annotation_guidelines} fueron adecuadas. Tambi\'en verifica si el proceso dise\~nado logra generar
anotaciones consistentes entre distintos anotadores. Una alta concordancia respalda la validez del proceso de anotaci\'on
planteado.

Adicionalmente, se quiere evaluar la calidad que tendr\'ia un anotador autom\'atico de prop\'osito general. Para esto, se 
emplea \emph{ChatGPT}~3.5 como anotador autom\'atico, y se comparan sus anotaciones con el corpus final.

\subsubsection{Escenario II}
\subsection{Corpus de Evaluaci\'on}
El corpus utilizado para la experimentaci\'on corresponde al conjunto de 150 textos con anotaciones de g\'enero y raza
generadas durante la construcci\'on del mismo, tal como se describe en la secci\'on~\ref{section:annotation_process}.

Espec\'ificamente, para el Escenario I se emplean las siguientes versiones del corpus:
\begin{itemize}
    \item Versi\'on 1: Conjunto de 150 textos con las anotaciones realizadas por el anotador 1.
    \item Versi\'on 2: Conjunto de 150 textos con las anotaciones realizadas por el anotador 2.
    \item Corpus Final: Versi\'on final del corpus de 150 textos, luego del proceso de mezcla y revisi\'on por parte 
    del anotador experto.
\end{itemize}

La comparaci\'on entre estas tres versiones del corpus permite estimar de manera realista la concordancia entre los anotadores 
no expertos y la concordancia de cada uno de ellos con la versi\'on final revisada. 

\subsection{Hiperpar\'ametros}
\subsection{Hardware}
Los experimentos fueron ejecutados en un equipo con las siguientes propiedades: 4 n\'ucleos de CPU \emph{Intel(R) Core(TM)} i7-6700K
@ 4.00GHz de velocidad con 8 MB de cach\'e y 16 GB de RAM DDR4 a una velocidad de 3200MHz.

El sistema operativo utilizado fue \emph{Windows 10 Pro}, versi\'on 21H2 y con una arquitectura de 64 bits.

\section{Resultados}\label{section:results}
A continuaci\'on se muestran los resultados obtenidos a partir de realizar los experimentos de la forma descrita en la 
secci\'on~\ref{section:experimental_framework}.

La Tabla~\ref{table:agreement_humans} y la Tabla~\ref{table:agreement_gpt} muestran los resultados del \textbf{Escenario~I}.
Estas tablas resumen los resultados de las m\'etricas de concordancia calculadas entre las distintas
versiones del corpus y el anotador autom\'atico \emph{ChatGPT}~3.5, respectivamente. En ellas, se 
calcula para los conjuntos $C_{g\acute{e}nero}$, $C_{raza}$ y $C_{g\acute{e}nero, raza}$ definidos
en la secci\'on~\ref{subsection:annotation_evaluation} las m\'etricas: media, varianza y desviaci\'on
est\'andar. Adem\'as, se calcula el \emph{macro-agreement} definido en dicha secci\'on.

\begin{table}[htpb]
    \centering
    \resizebox{\textwidth}{!}{
        \begin{tabular}{llccc}
        \toprule
        \textbf{M\'etrica de concordancia} &&\textbf{Ver. 1 - Ver. 2} & \textbf{Ver. 1 - Final} & \textbf{Ver. 2 - Final} \\
        \midrule
        \midrule
        \multirow{3}{*}{G\'enero} & $\mu C_{g\acute{e}nero}$                & 0.873 & 0.917 & 0.943\\
        \cmidrule{2-5}
                                  & $\sigma C_{g\acute{e}nero}$             & 0.324 & 0.268 & 0.228\\
        \cmidrule{2-5}
                                  & $\sigma^2 C_{g\acute{e}nero}$           & 0.105 & 0.072 & 0.052\\
        \midrule\midrule
        \multirow{3}{*}{Raza}     & $\mu C_{raza}$                          & 0.898 & 0.917 & 0.981\\
        \cmidrule{2-5}
                                  & $\sigma C_{raza}$                       & 0.279 & 0.256 & 0.124\\
        \cmidrule{2-5}
                                  & $\sigma^2 C_{raza}$                     & 0.078 & 0.066 & 0.015\\
        \midrule\midrule
        \multirow{4}{*}{General}  & $\mu C_{g\acute{e}nero, raza}$          & 0.870 & 0.903 & 0.956\\
        \cmidrule{2-5}
                                  & $\sigma C_{g\acute{e}nero, raza}$       & 0.263 & 0.232 & 0.173\\
        \cmidrule{2-5}
                                  & $\sigma^2 C_{g\acute{e}nero, raza}$     & 0.069 & 0.054 & 0.030\\
        \cmidrule{2-5}
                                  & Macro Agr                               & 0.886 & 0.917 & 0.962\\
        \bottomrule
        \end{tabular}}
    \caption{Resumen de m\'etricas de concordancia entre las distintas versiones del corpus.}
    \label{table:agreement_humans}
\end{table}

\begin{table}[htpb]
    \centering
    \resizebox{0.75\textwidth}{!}{
        \begin{tabular}{llc}
        \toprule
        \textbf{M\'etrica de concordancia} &&\textbf{ChatGPT - Final} \\
        \midrule
        \midrule
        \multirow{3}{*}{G\'enero} & $\mu C_{g\acute{e}nero}$                & 0.350\\
        \cmidrule{2-3}
                                  & $\sigma C_{g\acute{e}nero}$             & 0.455\\
        \cmidrule{2-3}
                                  & $\sigma^2 C_{g\acute{e}nero}$           & 0.207\\
        \midrule\midrule
        \multirow{3}{*}{Raza}     & $\mu C_{raza}$                          & 0.371\\
        \cmidrule{2-3}
                                  & $\sigma C_{raza}$                       & 0.475\\
        \cmidrule{2-3}
                                  & $\sigma^2 C_{raza}$                     & 0.225\\
        \midrule\midrule
        \multirow{4}{*}{General}  & $\mu C_{g\acute{e}nero, raza}$          & 0.329\\
        \cmidrule{2-3}
                                  & $\sigma C_{g\acute{e}nero, raza}$       & 0.352\\
        \cmidrule{2-3}
                                  & $\sigma^2 C_{g\acute{e}nero, raza}$     & 0.124\\
        \cmidrule{2-3}
                                  & Macro Agr                               & 0.361\\
        \bottomrule
        \end{tabular}}
    \caption{Resumen de m\'etricas de concordancia entre \emph{ChatGPT} y corpus final.}
    \label{table:agreement_gpt}
\end{table}


\section{Discusi\'on}
Los resultados presentados en la Tabla~\ref{table:agreement_humans} indican un alto grado de concordancia entre las 
anotaciones realizadas por los distintos anotadores humanos.

Espec\'ificamente, la concordancia entre los anotadores no expertos (\textbf{Ver.~1 - Ver.~2}) es bastante alta. La
media de las concordancias por anotaci\'on supera \textbf{0.85} para g\'enero, raza y la uni\'on de ambas en una \'unica categor\'ia,
lo que indica un alto grado de concordancia promedio. Adem\'as, la varianza y desviaci\'on est\'andar est\'an por debajo de 
\textbf{0.11} y \textbf{0.33} respectivamente, lo que demuestra poca dispersi\'on en los valores de concordancia por texto. El
\emph{macro-agreement} de \textbf{0.886}, refuerza la consistencia en t\'erminos generales. Todo esto sugiere que las directrices 
de anotaci\'on fueron adecuadas y permitieron obtener anotaciones consistentes entre anotadores no expertos.

Al comparar las anotaciones de los no expertos con la versi\'on final (\textbf{Ver.~1 - Final} y \textbf{Ver.~2 - Final}), se
observa una mejor\'ia en las m\'etricas. La media supera \textbf{0.90} en todos los casos, la varianza y desviaci\'on est\'andar
disminuyen por debajo de \textbf{0.08} y \textbf{0.28} respectivamente, y el \emph{macro-agreement} aumenta a \textbf{0.917} y 
\textbf{0.962}. Esto indica que la revisi\'on por el experto incrementa significativamente la calidad y consistencia de las 
anotaciones.

Los resultados de la Tabla~\ref{table:agreement_gpt} muestran m\'etricas de concordancia significativamente menores entre 
el anotador autom\'atico \emph{ChatGPT} y el corpus final. La media se encuentra por debajo de \textbf{0.38} en todos los casos,
indicando un bajo grado de coincidencia promedio. Adem\'as, la varianza supera \textbf{0.20} en los casos de g\'enero y raza, y 
\textbf{0.12} en la uni\'on. La desviaci\'on est\'andar supera \textbf{0.35} en todos los casos. Todo esto indica una alta 
dispersi\'on en los valores de concordancia por texto respecto a los obtenidos por los anotadores humanos. El 
\emph{macro-agreement} de apenas \textbf{0.361} evidencia la falta de consistencia del anotador autom\'atico. En conjunto, estas
m\'etricas proveen evidencia de que un anotador autom\'atico de prop\'osito general como \emph{ChatGPT} no es capaz de replicar
de manera confiable las anotaciones realizadas por los anotadores humanos.  
