\chapter{Descripci\'on del Corpus}\label{chapter:proposal}
En este cap\'itulo se propone un modelo de anotaci\'on de prop\'osito general de atributos protegidos
en textos. 
AQUI HAY QUE HACER UN PARRAFITO DESCRIBIENDO TODO LO QUE SE VA A HACER EN ESTE CAPITULO Y COMO SE VA A ORGANIZAR

\section{Esquema de anotaci\'on}\label{section:annotation_scheme}
%Hablar de que cada elemento a anotar es un texto, no una sola oracion.ok
%Hablar de cuales son los atributos protegidos que se anotaron. ok
%Hablar de los valores de cada atributo protegido y mas cosas relacionadas.ok/2
En el proceso de construcci\'on del corpus, cada elemento a anotar es un texto completo y no una oraci\'on individual. 
Este enfoque permite capturar el contexto completo en el que se producen las menciones a los atributos protegidos, 
lo cual es importante para poder realizar un an\'alisis m\'as profundo de (decir algo aqui como que es mejor asi para tratar los sesgos).

En este caso, los atributos protegidos que se van a anotar son el g\'enero y la raza.
Fueron seleccionados estos atributos no solo por su relevancia en diversas \'areas de estudio y aplicaci\'on, sino tambi\'en 
por la naturaleza de los textos que se van a anotar. Dado que las rese\~nas de pel\'iculas y series de televisi\'on
pueden contener un amplia gama de discusiones sobre actores, directores y personajes, es muy probable que se haga menci\'on a 
estos atributos.

Para el atributo g\'enero los valores a anotar son: ``Male'' para g\'enero masculino, ``Female'' para g\'enero femenino, 
``Male, Female'' cuando se detectan ambos g\'eneros y ``Null'' en caso de no detectar ninguno.

En cuanto a la raza, los valores a anotar son: ``White'' para raza blanca, ``Black'' para raza negra, ``Indian'' para 
personas originarias de la India, ``Arab'' para personas de origen \'arabe, que incluye a pa\'ises del medio oriente y el norte de 
\'Africa. Adem\'as se incorpora el valor ``Latino'' para personas de origen latinoamericano, ``Native American'' para personas 
originarias de los pueblos nativos de Am\'erica del Norte y ``Asian'' para personas asi\'aticas. Al igual que con el g\'enero, 
``Null'' indica que no se detecta ninguna raza en el texto. N\'otese que, en caso de ser necesario la anotaci\'on de la raza de un 
texto puede incluir m\'as de un valor.

\section{Proceso de anotaci\'on}\label{section:annotation_process}
El proceso de anotaci\'on comienza con la selecci\'on de los textos a anotar. Inicialmente se contaba con un conjunto de 70 textos 
extra\'idos de rese\~nas de pel\'iculas de ImDb\footnote{\url{https://www.kaggle.com/datasets/mantri7/imdb-movie-reviews-dataset}}, 
previamente anotados con g\'enero, producto de una tesis del grupo de investigaci\'on. A esta selecci\'on se a\~naden 80 textos m\'as 
extra\'idos de la misma fuente, con el objetivo de aumentar el tama\~no y la diversidad del corpus. Los nuevos textos se seleccionan 
de manera aleatoria.

Una vez seleccionados los textos, se procede a la anotaci\'on de los atributos protegidos. Esta etapa se divide en tres fases 
fundamentales:

\begin{enumerate}
    \item Anotaci\'on exhaustiva de ambos atributos protegidos por parte de dos anotadores no expertos en la prolem\'tica. 
    Generando as\'i dos anotaciones independientes. Los anotadores tienen permitido intercambiar opiniones y consultar dudas
    con un anotador experto, pero no deben discutir las oraciones espec\'ificas que anotan.
    \item Mezcla de las anotaciones independientes. En caso de detectarse conflicto, un anotador experto se encarga de escoger la 
    anotaci\'on m\'as acertada. Esta etapa puede ser asistida por \emph{scripts} de mezcla que detecten y resalten autom\'aticamente
    los conflictos.
    \item Revisi\'on del resultado final por parte de un anotador experto, en busca de posibles errores e inconsistencias.
\end{enumerate}

Luego de estas tres fases, el conjunto de textos anotados manualmente y revisados constituye el corpus, el cual debe ser evaluado 
como se describe en la secci\'on~\ref{subsection:annotation_evaluation}.
\subsection{Evaluaci\'on de la anotaci\'on}\label{subsection:annotation_evaluation}



\subsection{Directrices de anotaci\'on}
%Hablar que para anotar cierto atributo no hacer falta que el texto lo mencione explicitamente, sino que se puede inferir
%Ejemplo las entidades nombradas

\subsection{Herramientas de anotaci\'on}
%Describir los scripts que se hicieron para anotar, mezclar y todo eso


\section{Estad\'isticas del Corpus}


\section{Baselines}

\subsection{Detalles de implementaci\'on}

