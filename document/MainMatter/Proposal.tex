\chapter{Descripci\'on del Corpus}\label{chapter:proposal}
En este cap\'itulo se propone una metodolog\'ia de anotaci\'on de prop\'osito general de atributos protegidos
en textos. 
AQUI HAY QUE HACER UN PARRAFITO DESCRIBIENDO TODO LO QUE SE VA A HACER EN ESTE CAPITULO Y COMO SE VA A ORGANIZAR

\section{Esquema de anotaci\'on}
%Hablar de que cada elemento a anotar es un texto, no una sola oracion.ok
%Hablar de cuales son los atributos protegidos que se anotaron. ok
%Hablar de los valores de cada atributo protegido y mas cosas relacionadas.
En el proceso de construcci\'on del corpus, cada elemento a anotar es un texto completo y no una oraci\'on individual. 
Este enfoque permite capturar el contexto completo en el que se producen las menciones a los atributos protegidos, 
lo cual es importante para poder realizar un an\'alisis m\'as profundo de (decir algo aqui como que es mejor asi para tratar los sesgos).

En este caso, los atributos protegidos que se van a anotar son el g\'enero y la raza.
Fueron seleccionados estos atributos no solo por su relevancia en diversas \'areas de estudio y aplicaci\'on, sino tambi\'en 
por la naturaleza de los textos que se van a anotar. Dado que las rese\~nas de pel\'iculas y series de televisi\'on
pueden contener un amplia gama de discusiones sobre actores, directores y personajes, es muy probable que se haga menci\'on a 
estos atributos.

Para el atributo g\'enero los valores a anotar son: "Male" para g\'enero masculino, "Female" para g\'enero femenino, 
"Male, Female" cuando se detectan ambos g\'eneros y "Null" en caso de no detectar ninguno.

En cuanto a la raza, los valores a anotar son: "White" para raza blanca, "Black" para raza negra, "Indian" para 
personas originarias de la India, "Arab" para personas de origen \'arabe, que incluye a pa\'ises del medio oriente y el norte de \'Africa.
Adem\'as se incorpora el valor "Latino" para personas de origen latinoamericano, "Native American" para personas originarias de 
los pueblos nativos de Am\'erica del Norte y "Asian" para personas asi\'aticas. Al igual que con el g\'enero, "Null" indica que no se detecta
ninguna raza en el texto. N\'otese que, en caso de ser necesario la anotaci\'on de la raza de un texto puede incluir m\'as de un valor.

\section{Proceso de anotaci\'on}

\subsection{Evaluaci\'on de la anotaci\'on}

\subsection{Directrices de anotaci\'on}
%Hablar que para anotar cierto atributo no hacer falta que el texto lo mencione explicitamente, sino que se puede inferir
%Ejemplo las entidades nombradas

\subsection{Herramientas de anotaci\'on}
%Describir los scripts que se hicieron para anotar, mezclar y todo eso


\section{Estad\'isticas del Corpus}

\section{Baselines}

\subsection{Detalles de implementaci\'on}

