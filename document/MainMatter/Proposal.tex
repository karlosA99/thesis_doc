\chapter{Descripci\'on del Corpus}\label{chapter:proposal}
En este cap\'itulo se propone una metodolog\'ia de anotaci\'on de prop\'osito general de atributos protegidos
en textos. 
AQUI HAY QUE HACER UN PARRAFITO DESCRIBIENDO TODO LO QUE SE VA A HACER EN ESTE CAPITULO Y COMO SE VA A ORGANIZAR

\section{Esquema de anotaci\'on}
%Hablar de que cada elemento a anotar es un texto, no una sola oracion.
%Hablar de cuales son los atributos protegidos que se anotaron
%Hablar de los valores de cada atributo protegido y mas cosas relacionadas
En el proceso de construcci\'on del corpus, cada elemento a anotar es un texto completo y no una oraci\'on individual. 
Este enfoque permite capturar el contexto completo en el que se producen las menciones a los atributos protegidos, 
lo cual es importante para poder realizar un an\'alisis m\'as profundo de (decir algo aqui como que es mejor asi para los sesgos).

\section{Proceso de anotaci\'on}

\subsection{Evaluaci\'on de la anotaci\'on}

\subsection{Directrices de anotaci\'on}
%Hablar que para anotar cierto atributo no hacer falta que el texto lo mencione explicitamente, sino que se puede inferir
%Ejemplo las entidades nombradas

\subsection{Herramientas de anotaci\'on}
%Describir los scripts que se hicieron para anotar, mezclar y todo eso


\section{Estad\'isticas del Corpus}

\section{Baselines}

\subsection{Detalles de implementaci\'on}

