\chapter{Estado del Arte}\label{chapter:state-of-the-art}

\begin{itemize}
    \item Empezar hablando de los algoritmos de ML y hablar de su uso en la actualidad.
    \item Luego hablar de bias y fairness en general.
    \item Hablar de las preocupaciones que existen por el bias y fairness en los algoritmos de ML.
\end{itemize}

Los algoritmos de aprendizaje autom\'atico han permeado pr\'acticamente todos los aspectos de la vida moderna,
desde proporcionar recomendaciones de pel\'iculas y facilitar compras en l\'inea hasta influir en b\'usquedas en la web
y sugerir conexiones emocionales con otras personas. Este fen\'omeno se extiende incluso a escenarios m\'as riesgosos,
como el diagn\'ostico y tratamiento m\'edico, donde el uso de estos algoritmos ha experimentado un notable aumento.
La versatilidad de estas herramientas abarca diversas \'areas, como la optimizaci\'on de procesos empresariales, la 
mejora de la eficiencia de los sistemas de transporte \cite{autonomous_driving}, y la personalizaci\'on de servicios en 
sectores como el financiero y el comercio.

A medida que estos algoritmos se aplican con mayor frecuencia en \'ambitos m\'as cr\'iticos, como
pr\'estamos bancarios \cite{fairness_def}, evaluaci\'on de riesgos de salud, contrataci\'on, evaluaci\'on de desempe\~no laboral y
justicia penal \cite{compas}, se genera una creciente preocupaci\'on acerca de su capacidad para mantener de manera involuntaria 
sesgos sociales y prejuicios hist\'oricos. 

\section{Justicia y equidad en modelos de aprendizaje autom\'atico}

Adentrarse en el terreno de las definiciones de equidad en el contexto de los modelos de aprendizaje de m\'aquinas nos lleva 
a un panorama complejo y en constante evoluci\'on. A d\'ia de hoy, no existe una definici\'on \'unica y precisa
de lo que constituye la equidad en este \'ambito. La implementaci\'on de algoritmos en la toma de decisiones automatizada ha desatado 
debates acerca de c\'omo conceptualizar y medir tanto la equidad como la justicia. Estos conceptos no solo involucran concideraciones 
t\'ecnicas, sino que tambi\'en se ven influidos por matices culturales y dilemas \'eticos.

\subsection{Definiciones de equidad}
Las diversas perspectivas sobre la equidad pueden agruparse en tres categor\'ias principales: a nivel de Grupos, Subgrupos y a nivel Individual. 
A continuaci\'on se presentan algunas de las definiciones de equidad m\'as relevantes a nivel de grupos:

\begin{itemize}
    \item \textbf{Demographic Parity}: \cite{fairness_def}
    \item \textbf{Equal Opportunity}: \cite{fairness_def}
    \item \textbf{Equalized Odds}: \cite{fairness_def}
\end{itemize}

Una m\'etrica muy relevante a nivel de subgrupos es:

\begin{itemize}
    \item \textbf{Subgroup Fairness}: \cite{subgroup_fairness}
\end{itemize}

Finalmente, a nivel individual tenemos:

\begin{itemize}
    \item \textbf{Fairness Through Awareness}: \cite{fair_awareness}
    \item \textbf{Fairness Through Unawareness}: \cite{counterfactual}
    \item \textbf{Conterfactual fairness}: \cite{counterfactual}
\end{itemize}

\section{Hablar de las fuentes de bias}
\begin{itemize}
    \item hablar de los tres tipos mas comunes de sesgos que se mencionan en "Julia Stoyanovich, Bill Howe, and HV Jagadish. Responsible data management. Proceedings of the VLDB
    Endowment, 13(12):3474–3488, 2020.": ¿Hacer enfasis en el sesgo de la data?
    \begin{itemize}
        \item pre-existente: proviene de la data
        \item tecnologico: proviene de los algoritmos
        \item emergente: proviene de la interaccion con los usuarios
    \end{itemize}
    \item hablar algo de las 23 fuentes mas comunes de bias 
    que se mencionan en "Ninareh Mehrabi, Fred Morstatter, Nripsuta Saxena, Kristina Lerman, and Aram Galstyan. A survey on bias and
    fairness in machine learning. ACM Comput. Surv., 54(6), jul 2021."
\end{itemize}

\section{Hablar de Casos reales donde se haya detectado sesgos}

    \begin{itemize}
        \item Hablar del caso COMPAS, MEPS y Adult. Ver los paper que se mencionan en el Survey1 en pag 8
    \end{itemize}

\section{Hablar de los datasets estudiados}
    \begin{itemize}
        \item Poner la tabla comparativa de los datasets
        \item Decir algunos datos interesantes de cada uno
        \item ¿Empezar a hablar de xq se escogio imdb?
    \end{itemize}

\section{Discusi\'on}

