\begin{resumen}
	%Parrafo 1
	% -Empezar hablando en general del incrementod de las aplicaciones de ML
	% -Decir que a medida que se emplean estos sistemas en escenarios cada vez mas criticos 
	%se ha despertado la preocupacion acerca de la equidad e imparcialidad de los mismos
	% -Decir que se han desarrolladoe estudios con el objetivo de investigar acerca de los posibles 
	%sesgos en los sistemas de ML, y en fecto, se han encontrado sistemas que no son justos con determinado
	%grupo de personas
	% -Decir que sin embargo las tecnicas para detectar y mitigar estos sesgos necesitan de datasets
	%anotados con atributos protegidos, y que la mayoria de los datasets disponibles tienen estructura tabular
	
	%Parrafo 2
	% -Decir que esta tesis propone el diseño y validacion de un corpus de datos no tabulares, con anotaciones de atribuitos
	%protegidos y toma de decisiones en textos de dominio general.
	% -Decir que se disena un modelo de anotacion de proposito general que busca maximizar la calidad y consistencia de las anotaciones
	%- Decir que se construye un corpus de textos de dominio general anotados segun el modelo anterior
	% -Decir que se evalua la efectividad del modelo de anotacion en ser aprendido automaticamente, 
	%mediante la implementacion de un sistema de extraccion automatica de las anotaciones, utilizando el
	%corpus construido como principal escenario de entrenamiento y evaluacion.
	% -Terminar diciendo algo de que los resultados alcanzados demuestran...

\end{resumen}

\begin{abstract}
	Resumen en inglés
\end{abstract}