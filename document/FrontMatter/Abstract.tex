\begin{resumen}
	%Parrafo 1
	% -Empezar hablando en general del incrementod de las aplicaciones de ML
	% -Decir que a medida que se emplean estos sistemas en escenarios cada vez mas criticos 
	%se ha despertado la preocupacion acerca de la equidad e imparcialidad de los mismos
	% -Decir que se han desarrolladoe estudios con el objetivo de investigar acerca de los posibles 
	%sesgos en los sistemas de ML, y en fecto, se han encontrado sistemas que no son justos con determinado
	%grupo de personas
	% -Decir que sin embargo las tecnicas para detectar y mitigar estos sesgos necesitan de datasets
	%anotados con atributos protegidos, y que la mayoria de los datasets disponibles tienen estructura tabular
	
	%Parrafo 2
	% -Decir que esta tesis propone el diseño y validacion de un corpus de datos no tabulares, con anotaciones de atribuitos
	%protegidos y toma de decisiones en textos de dominio general.
	% -Decir que se disena un modelo de anotacion de proposito general que busca maximizar la calidad y consistencia de las anotaciones
	%- Decir que se construye un corpus de textos de dominio general anotados segun el modelo anterior
	% -Decir que se evalua la efectividad del modelo de anotacion en ser aprendido automaticamente, 
	%mediante la implementacion de un sistema de extraccion automatica de las anotaciones, utilizando el
	%corpus construido como principal escenario de entrenamiento y evaluacion.
	% -Terminar diciendo algo de que los resultados alcanzados demuestran...

	En los \'ultimos a\~nos ha habido un incremento sustancial en el uso de algoritmos de aprendizaje autom\'atico, emple\'andose en
	escenarios cada vez m\'as cr\'iticos. A medida que estos sistemas se utilizan para la toma de decisiones sensibles, ha surgido 
	la preocupaci\'on por la equidad e imparcialidad de los mismos. Diversos estudios han investigado la presencia de posibles sesgos
    en modelos de aprendizaje autom\'atico, y en efecto, se han encontrado sistemas que no son justos con determinados grupos de personas.
	A ra\'iz de esto, se han desarrollado t\'ecnicas para detectar y mitigar estos sesgos, las cuales necesitan de \emph{datasets} anotados
	con atributos protegidos. La mayor\'ia de los \emph{datasets} disponibles poseen una estructura tabular, lo cual limita su aplicabilidad
	para el an\'alisis de sesgos en tareas donde se requieren datos no estructurados como texto, audio e im\'agenes.
	
	Esta tesis propone el dise\~no y validaci\'on de un corpus de datos no tabulares, con anotaciones de atributos protegidos y toma de decisiones, 
	en textos de dominio general. Se dise\~na un modelo de anotaci\'on de prop\'osito general que busca maximizar la calidad y consistencia de las 
	anotaciones. Se construye un corpus de textos anotados seg\'un este modelo. Para evaluar la efectividad del modelo de anotaci\'on en ser 
	aprendido autom\'aticamente, se implementa un sistema de extracci\'on autom\'atica de las anotaciones, utilizando el corpus generado como 
	escenario de entrenamiento. Los resultados alcanzados demuestran la viabilidad del corpus y el equema de anotaci\'on propuestos para asistir 
	en el an\'alisis de sesgos.



\end{resumen}

\begin{abstract}
	Resumen en inglés
\end{abstract}