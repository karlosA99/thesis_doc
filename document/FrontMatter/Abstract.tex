\begin{resumen}
	%Parrafo 1
	% -Empezar hablando en general del incrementod de las aplicaciones de ML
	% -Decir que a medida que se emplean estos sistemas en escenarios cada vez mas criticos 
	%se ha despertado la preocupacion acerca de la equidad e imparcialidad de los mismos
	% -Decir que se han desarrolladoe estudios con el objetivo de investigar acerca de los posibles 
	%sesgos en los sistemas de ML, y en fecto, se han encontrado sistemas que no son justos con determinado
	%grupo de personas
	% -Decir que sin embargo las tecnicas para detectar y mitigar estos sesgos necesitan de datasets
	%anotados con atributos protegidos, y que la mayoria de los datasets disponibles tienen estructura tabular
	
	%Parrafo 2
	% -Decir que esta tesis propone el diseño y validacion de un corpus de datos no tabulares, con anotaciones de atribuitos
	%protegidos y toma de decisiones en textos de dominio general.
	% -Decir que se disena un modelo de anotacion de proposito general que busca maximizar la calidad y consistencia de las anotaciones
	%- Decir que se construye un corpus de textos de dominio general anotados segun el modelo anterior
	% -Decir que se evalua la efectividad del modelo de anotacion en ser aprendido automaticamente, 
	%mediante la implementacion de un sistema de extraccion automatica de las anotaciones, utilizando el
	%corpus construido como principal escenario de entrenamiento y evaluacion.
	% -Terminar diciendo algo de que los resultados alcanzados demuestran...

	En los \'ultimos a\~nos ha habido un incremento sustancial en el uso de algoritmos de aprendizaje autom\'atico, emple\'andose en
	escenarios cada vez m\'as cr\'iticos. A medida que estos sistemas se utilizan para la toma de decisiones sensibles, ha surgido 
	la preocupaci\'on por la equidad e imparcialidad de los mismos. Diversos estudios han investigado la presencia de posibles sesgos
    en modelos de aprendizaje autom\'atico, y en efecto, se han encontrado sistemas que no son justos con determinados grupos de personas.
	A ra\'iz de esto, se han desarrollado t\'ecnicas para detectar y mitigar estos sesgos, las cuales necesitan de \emph{datasets} anotados
	con atributos protegidos (g\'enero, raza, religi\'on, etc.). La mayor\'ia de los \emph{datasets} anotados con atributos protegidos,
	poseen una estructura tabular, lo cual limita su aplicabilidad para el an\'alisis de sesgos en tareas donde se 
	requieren datos no estructurados como texto, audio e im\'agenes.
	
	Esta tesis propone el dise\~no y validaci\'on de un corpus de datos no tabulares, con atributos protegidos anotados y toma de 
	decisiones, en textos de rese\~nas de pel\'iculas. Se dise\~na un esquema de anotaci\'on de prop\'osito general, que busca maximizar 
	la calidad y consistencia de las anotaciones. Se construye un corpus de textos anotados seg\'un este esquema. Para evaluar la 
	efectividad del esquema de anotaci\'on en ser aprendido autom\'aticamente, se implementa un sistema de extracci\'on autom\'atica de 
	las anotaciones, utilizando el corpus generado como escenario de entrenamiento. Los resultados alcanzados demuestran la viabilidad 
	del corpus y el esquema de anotaci\'on propuestos para asistir en el an\'alisis de sesgos.



\end{resumen}

\begin{abstract}
	In recent years, there has been a substantial increase in the use of machine learning algorithms, being used in increasingly critical scenarios.
	As these systems are used for sensitive decision-making, concerns about their fairness and impartiality have arisen.
	Various studies have investigated the presence of potential biases in machine learning models, and indeed, systems that are not fair to certain 
	groups of people have been found. As a result, techniques have been developed to detect and mitigate these
	biases, which require annotated datasets with protected attributes (gender, race, religion, etc.). 
	Most of the available datasets, annotated with protected attributes, have a tabular structure, which limits their applicability for the
	analysis of biases in tasks where unstructured data such as text, audio and images are required.

	This thesis proposes the design and validation of a non-tabular dataset with annotated protected attributes and decision-making in movie review texts.
	A general-purpose annotation scheme is designed to maximize the quality and consistency of the annotations.
	A corpus of annotated texts is built according to this scheme.
	To evaluate the effectiveness of the annotation scheme in being learned automatically, an automatic extraction system of the annotations is implemented, 
	using the generated corpus as a training scenario.
	The results demonstrate the feasibility of the proposed corpus and annotation scheme to assist in bias analysis.
\end{abstract}