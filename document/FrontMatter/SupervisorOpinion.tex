\begin{opinion}

En la actualidad, los algoritmos de aprendizaje automático están siendo aplicados en disímiles áreas de la vida humana. En particular, su incorporación a tareas de toma de decisiones de alto riesgo ha dirigido la atención de muchos investigadores hacia una nueva interrogante: ¿estarán siendo ``justos'' los algoritmos de aprendizaje automático al tomar sus decisiones? El concepto de justicia o equidad se interpreta en este contexto como la ausencia de cualquier prejuicio o favoritismo hacia un individuo o grupo basado en sus características inherentes o adquiridas. El peligro fundamental de ignorar la interrogante planteada anteriormente radica en que los métodos de aprendizaje automático podrían no solo reflejar los sesgos presentes en nuestra sociedad, sino que también podrían amplificarlos. Resolver problemas de forma justa debería convertirse en un estándar en todos los contextos en los que es aplicable. En este marco se desarrolla la tesis de licenciatura de Karlos Alejandro Alfonso Rodríguez, con quien pude trabajar este último año en el diseño y validación de un corpus para auxiliar en desarrollo de técnicas de cuantificación y mitigación de sesgos.

La propuesta de Karlos consiste en un corpus de textos con anotaciones de atributos protegidos y toma de decisiones. Los atributos protegidos que se anotan son: género y raza. Estos atributos son los más demandados en el área de estudio y, sin embargo, que pudo comprobar con el estudio del estado del arte que escasean recursos que los tengan anotados manualmente a la par que alguna otra característica que denote calidad o decisión. Los textos de reseñas de películas de IMDb cumplían parcialmente esas restricciones, de ahí que haya sido tomada como base. Adicionalmente, los textos de dicha fuente incluyen referencias a entidades nombradas y pronombres, que enriquecen las formas de relacionar el texto a atributos protegidos. Esto se traduce en que los modelos de anotación requerirán acceder a información externa al corpus o de dominio general para producir soluciones robustas. Las contribuciones de la tesis se pueden resumir en tres elementos: una metodología de anotación, un corpus anotado y varios modelos de anotación automática que sirven como punto de partida para futuras investigaciones. Los resultados obtenidos avalan la efectividad del proceso de anotación y muestran la diferencia de calidad entre anotadores humanos y automáticos. El resultado final es una propuesta teórica, respaldada por un prototipo computacional, que demuestra que el estudiante posee las habilidades necesarias para aplicar en la práctica sus conocimientos. 

Durante el desarrollo de esta investigación, Karlos ha tenido que asimilar por su cuenta conocimientos de diversas áreas, como procesamiento de lenguaje natural e inteligencia artificial en general. Además, ha tenido que estudiar en profundidad un campo de investigaciones tan novedoso y variado como es el análisis de sesgos en algoritmos de aprendizaje automático. El proceso de investigación e implementación desarrollado por Karlos queda recogido en un documento de tesis que avala además sus habilidades para llevar a buen término una investigación científica con la formalidad que el campo requiere. El estudiante ha demostrado así no solo dominio técnico del área, sino además capacidad de organización.  Todo esto lo han realizado a la par de las actividades docentes como estudiante de pregrado.

Conocí a Karlos cuando cursaba su primer año de la carrera, donde le impartí clases de la asignatura Programación. Coincidimos varias veces más a lo largo de la carrera, pero no fue hasta hace un año que me pidió que trabajáramos juntos en su tesis de diploma. Gracias a eso pude descubrir lo satisfactorio de trabajar en equipo con Karlos. Muchas veces creí que terminar una tarea en tiempo sería un trabajo abrumador, pero Karlos adoptó la tarea con la mayor tranquilidad posible y para mi grata sorpresa en efecto pudo obtener los resultados en tiempo. La habilidad para encarar los problemas sin miedo es ciertamente digna de admirar, y, en mi opinión, Karlos rebosa de ella. Con este último ejercicio, Karlos demostró haber adquirido la madurez necesaria para desarrollar proyectos de alta complejidad con calidad y esmero. Como tutor, estoy complacido por los resultados obtenidos y por el trabajo realizado por Karlos, que superó todos los desafíos, incluido el tener que comenzar la tesis con una supervisión remota. Tengo plena confianza en que ha de cosechar las recompensas por todo el empeño que ha puesto en sus estudios y en la investigación, y que ejercerá como una excelente profesional.

    
\vspace{1cm}

\begin{center}
\emph{MSc. Juan Pablo Consuegra Ayala} \hspace{15pt} \emph{Dr. Suilan Estévez Velarde}\\
    Facultad de Matemática y Computación \\
    Universidad de La Habana    
\end{center}

\end{opinion}